%%%%%%%%%%%%%%%%%%%%%%%%%%%%%%%%%%%%%%%%%%%%%%%%%%%%%%%%%%%%%%%%%%%%%%%%%%%
% 
% Szablon manuskryptu publikacji do czasopisma anglojęzycznego.
%          
% Autor:  Paweł Kłeczek   (pkleczek@agh.edu.pl)
%             AGH (Kraków), WEAIiIB, KAiR
% 
% Ostatnia modyfikacja: 2018-11-10
%
%%%%%%%%%%%%%%%%%%%%%%%%%%%%%%%%%%%%%%%%%%%%%%%%%%%%%%%%%%%%%%%%%%%%%%%%%%%%

\documentclass[pdftex,11pt,a4paper]{article}

%------------------------------------------------------------------------------
% Rozmiary "obszarów roboczych", skład tekstu.
% SEC: rozmiary "obszarów roboczych", skład tekstu
%------------------------------------------------------------------------------

\usepackage{geometry}  % ustawienie rozmiaru marginesów, nagłówka i stopki itp.
\geometry{lmargin=20mm, rmargin=20mm, tmargin=15mm, bmargin=20mm, foot=10mm, head=10mm}

\usepackage{indentfirst}  % domyślne wcinanie pierwszego akapitu w bloku tekstu

%------------------------------------------------------------------------------
% Kodowanie znaków, lokalizacja.
% SEC: kodowanie znaków, lokalizacja
%------------------------------------------------------------------------------

\usepackage[utf8]{inputenc}  % wybierz kodowanie UTF-8
\usepackage{mathptmx}  % używaj czcionki Times jako domyślnej dla tekstu
\usepackage[T1]{fontenc}  % umożliwia wstawianie znaków z kodowania 8-bitowego (domyślnie jest 7-bitowe, przez co część znaków musi być wstawiana za pomocą kombinacji np. \l{} - "ł", \"{o} - "ö"; zob. https://tex.stackexchange.com/a/677/44391)
\usepackage[english]{babel}  % ustaw angielską lokalizację

%------------------------------------------------------------------------------
% Podstawowe pakiety.
% SEC: podstawowe pakiety
%------------------------------------------------------------------------------

\usepackage{color}  % STD
\usepackage[table,usenames,dvipsnames]{xcolor}

\usepackage[inline,shortlabels]{enumitem}  % zwiększa kontrolę użytkownika nad środowiskami `itemize` i `enumerate` (następca pakietu `enumerate`)

%------------------------------------------------------------------------------
% Tworzenie i konfiguracja tabel.
% SEC: tabele
%------------------------------------------------------------------------------

%\usepackage{hhline}
%\usepackage{longtable}
\usepackage{array}  % STD
\usepackage{tabularx}  % STD
\usepackage{multirow}  % umożliwia wstawienie tekstu zajmującego kilka wierszy [\multirow] - wymagane przez pakiet `makecell`
\usepackage{booktabs}  % poprawia estetykę tabel [\toprule, \midrule, \bottomrule, ...]
\usepackage{makecell}  % estetyczne, konfigurowalne formatowanie nagłówków tabeli [\thead]; umożliwia scalanie komórek [\makecell]
\usepackage[flushleft]{threeparttable}  % tabela z przypisami

% Definicje nowych typów kolumn w oparciu o typ `X` z pakietu `tabularx` (zob. https://en.wikibooks.org/wiki/LaTeX/Tables#The_tabularx_package)
\def\tabularxcolumn#1{m{#1}}  % spraw, aby kolumna typu X korzystała z `m` (`\parbox[c]`) zamiast z `p` (`parbox[t]`)
\newcolumntype{C}[1]{>{\hsize=#1\hsize\centering\arraybackslash}X}  % kolumna z wyśrodkowaną zawartością, o zadanej proporcji szerokości (zob. przykład z macierzą pomyłek)
\newcolumntype{Y}{>{\centering\arraybackslash}X}  % kolumna z wyśrodkowaną zawartością, bez precyzowania proporcji szerokości (zob. przykład z macierzą pomyłek)

\newlength{\multirowverticaloffset}
\setlength{\multirowverticaloffset}{2mm}

%------------------------------------------------------------------------------
% LaTeX (techniczne).
% SEC: LaTeX (techniczne)
%------------------------------------------------------------------------------

\usepackage{keyval,xparse}  % definiowanie nowych poleceń

\usepackage{cprotect}  % Pozwala na użycie środowiska `verbatim` w podpisach ilustracji, w tytułach sekcji itd. [\cprotect\caption{...}, \cprotect\section{...}]

\usepackage{courier}  % umożliwia korzystanie z czcionki Courier

\usepackage{footnote}  % lepsza kontrola przypisów w stopce

\usepackage{filecontents}  % emulowanie tworzenia pliku z danymi (np. pod kątem generowania wykresów) z poziomu kodu LaTeX-a

%------------------------------------------------------------------------------
% Matematyka.
% SEC: matematyka
%------------------------------------------------------------------------------

\usepackage{mathtools}  % STD
\usepackage{amsfonts}  % STD
\usepackage{amsmath}  % STD
\usepackage{amsthm}  % STD

%------------------------------------------------------------------------------
% Symbole.
% SEC: symbole
%------------------------------------------------------------------------------

\usepackage{textcomp}  % \textdegree (symbol stopni poza środowiskiem matematycznym)
\usepackage{gensymb}  % symbole matematyczne: \degree, \celsius, \micro, ...

%------------------------------------------------------------------------------
% Odnośniki.
% SEC: odnośniki
%------------------------------------------------------------------------------

\usepackage{url}  % definiuje polecenie `\url` służące do wyświetlania m.in. adresów e-mail, odnośników URL i ścieżek do plików - z możliwością podziału długiego adresu (domyślnie LaTeX nie dzieli takich ścieżek)
\renewcommand\UrlFont{\rmfamily\itshape}  % wyświetlaj adresy URL za pomocą czcionki Times z italikami

\usepackage{nameref}  % umożliwia wstawianie odnośników do sekcji z wyświetlaną nazwą sekcji 

\usepackage{hyperref}  % funkcjonalność "klikalnych" odnośników w dokumentach PDF
%\usepackage[hidelinks]{hyperref}  % wersja bez ramek wokół odnośników

%------------------------------------------------------------------------------
% Ustawienia ilustracji i ich podpisów.
% SEC: ilustracje i podpisy
%------------------------------------------------------------------------------

\usepackage{graphicx}  % STD

\usepackage{float}  % STD

\usepackage[font=small, labelfont=bf, labelsep=period]{caption}  % formatowanie podpisów pod ilustracjami (mała czcionka, wytłuszczony "identyfikator" zakończony kropką - a nie dwukropkiem)
%\captionsetup[subfigure]{labelfont=md}  % możliwość konfiguracji podpisów pod zadanym typem ilustracji/tabel...
%\captionsetup{subrefformat=parens}  % spraw, aby domyślnie odnośniki do ilustracji w grupie otaczane były nawiasami okrągłymi
\captionsetup{margin=2cm, singlelinecheck=on}  % singlelinecheck=on - podpis zawsze wyśrodkowany, gdy mieści się w jednej linii

\usepackage{subcaption}  % tworzenie grup ilustracji [\subfigure]
%\usepackage{subfig}  % obecnie zaleca się używać pakietu `subcaption` [2018-11-10]
\usepackage{afterpage}  % wymuszenie wyświetlenia wszystkich ilustracji i kontynuowanie wyświetlania tekstu na tej samej stronie [\afterpage{\clearpage}] https://tex.stackexchange.com/questions/88657/clearpage-without-pagebreak

% The new graphbox package introduces a new (vertical) align key which can be used in \includegraphics options.
\usepackage{graphbox}

% --------------------------------------

\newlength{\groupwidth}  % szerokość środowiska `figure`
\newlength{\picwidth}  % szerokość obszaru pojedynczego środowiska `subfigure`
\newlength{\imgwidth}  % szerokość grafiki w `subfigure`
\newlength{\imgheight}  % wysokość grafiki w `subfigure`

\newlength{\subfigshsep}  % odstęp między ilustracjami w poziomie
\newlength{\subfigshsepdef}  % odstęp między ilustracjami w poziomie (predefiniowany)
\newlength{\subfigsvsep}  % odstęp między ilustracjami w pionie
\newlength{\subfigsvsepdef}  % odstęp między ilustracjami w pionie (predefiniowany)
% [NOTE] Użycie dwóch typów długości - o wartościach modyfikowanych na bieżąco
%	i o wartościach predefiniowanych - pozwala zachować spójność dokumentu.

\setlength{\subfigshsepdef}{0.2cm}
\setlength{\subfigsvsepdef}{0.3cm}

% Domyślnie `figure` zajmuje 90% szerokości tekstu.
\setlength{\groupwidth}{0.9\textwidth}

% Konfiguracja środowiska `fbox` (do wyświetlania ramek)
\setlength{\fboxsep}{0pt}
\setlength{\fboxrule}{0.5pt}

%------------------------------------------------------------------------------
% Ustawienia parametrów odstępów i rozmiarów dla tytułów (pod)rozdziałów, czcionki, bibliografii, ilustracji i list - pozwalają "scieśnić" dokument.
% SEC: odstępy
%------------------------------------------------------------------------------

\usepackage{titlesec}

% Zmniejsz odstępy wokół list i między punktami listy.
\setlist{topsep=2pt,itemsep=0.1ex,partopsep=1ex,parsep=0.5ex,leftmargin=0.75cm}
%\setlength{\itemindent}{-1.5cm}

\newlength{\interpointgap}
\setlength{\interpointgap}{0.1em}

\setlength\parindent{0.4cm}

% Środowisko `enumerate` wymaga osobnej konfiguracji...
\setlist[enumerate]{topsep=2pt,itemsep=-1ex,partopsep=1ex,parsep=0.5ex}

% Zdefiniuj "ciasne" środowisko `enumerate` - z minimalnymi odstępami.
\newenvironment{tight_enumerate}{
\begin{enumerate}
	\setlength{\topsep}{4pt}
	\setlength{\itemsep}{4pt}
	\setlength{\parskip}{0pt}
}{\end{enumerate}}

% --------------------------------------

% Zmniejsz odstępy między literami.
%\usepackage[letterspace=-40]{microtype}

% Zmniejsz interlinię.
%\linespread{0.9}\selectfont

% --------------------------------------

% Zmniejsz odstęp między wpisami bibliograficznymi.
%\setlength\bibitemsep{0.4\itemsep}  % tylko dla pakietu `biblatex`

% --------------------------------------

% Zmniejsz odstępy między ilustracją i tekstem.
\setlength{\textfloatsep}{3pt}

% Zmniejsz odstęp między dwiema ilustracjami.
\setlength{\floatsep}{3pt}

% Zmniejsz odstępy powyżej i poniżej ilustracji.
\setlength{\intextsep}{3pt}

% Zmniejsz odstępy między grafiką a podpisem.
\setlength{\abovecaptionskip}{3pt}

% --------------------------------------

% Zmień formatowanie tytułów (pod)rozdziałów.

\titleformat{\section}{\bfseries\large}{\thesection}{1em}{}
\titlespacing{\section}{0pt}{1em}{0.5em}

%\titleformat{\subsection}{\bfseries\normalsize}{\thesubsection}{1em}{}
%\titleformat{\subsubsection}{\bfseries\normalsize}{\thesubsubsection}{1em}{}

%------------------------------------------------------------------------------
% Configure TikZ/PGFplots
% SEC: TikZ / PGFplots
%------------------------------------------------------------------------------

\usepackage{tikz}
\usepackage{pgfplots}
\usepackage{tikzscale}

\usetikzlibrary{calc,shapes,shadows,arrows,positioning,graphs}
\tikzset{>=latex}

\pgfplotsset{compat=1.8}
\pgfplotsset{minor grid style={dashed,black!15!white}}	
\pgfplotsset{major grid style={black!35!white}}	

\usepgfplotslibrary{groupplots}

%------------------------------------------------------------------------------
% Configure siunitx
% SEC: siunitx
%------------------------------------------------------------------------------

\usepackage{siunitx}  % wyświetlanie sformatowanych liczb (m.in. z jednostkami)

\sisetup{%
	range-phrase = {--},  % używaj półpauzy do oznaczenia zakresu
	range-units = single,  % nie powtarzaj jednostek przy zakresach
	group-digits = true,  % grupuj cyfry tysiącami
	group-separator = {,},  % separator grup zgodny z zapisem w USA
	output-decimal-marker = {.},  % separator części ułamkowej zgodny z zapisem w USA
}

\DeclareSIUnit\pixel{px}  % zdefiniuj własną jednostkę - piksel

%------------------------------------------------------------------------------
% Ustawienia bibliografii.
% SEC: bibliografia
%------------------------------------------------------------------------------

\usepackage[
	backend=biber,
	style=numeric,
	sorting=none,
	%
	% Zastosuj styl wpisu bibliograficznego właściwy językowi publikacji.
	language=autobib,
	autolang=other,
	%
	urldate=iso8601,
	% Nie dodawaj numerów stron, na których występuje cytowanie.
	backref=false,
	isbn=true,
	url=false,
	%
	% Ustawienia związane z polskimi normami dla bibliografii.
	maxbibnames=3,
]{biblatex}

\addbibresource{pub_sample_bibliography.bib}  % dodaj literaturę z zadanego pliku
\AtEveryBibitem{\clearfield{note}}  % usuń zawartość pól `note` z pozycji bibliograficznych

\usepackage{csquotes}
% NOTE: Ponieważ `csquotes` nie posiada polskiego stylu, można skorzystać z mocno zbliżonego stylu chorwackiego.
\DeclareQuoteAlias{croatian}{polish}

% Zmień sposób formatowania kombinacji "tom + numer + elektroniczne ID" (przydatne przy niektórych czasopismach).
%\renewbibmacro*{volume+number+eid}{%
%	\printfield{volume}%
%	\setunit{\addcomma\space}%
%	\printfield{number}%
%	\setunit{\addcomma\space}%
%	\printfield{eid}}

%------------------------------------------------------------------------------
% Konfiguracja pakietu `fp`
% SEC: fp
%------------------------------------------------------------------------------

\usepackage{fp}

% [NOTE] Polecenie `\FPuse` służy do wyliczania wartości wyrażenia "w miejscu" (tj. żeby od razu użyć wyliczonej wartości np. w innym środowisku).
\newcommand\FPuse[1]{\FPeval{\result}{#1}\result}


% [NOTE] Zdefiniuj polecenie `\numperc` służące do wyświetlania danego ułamka jako procentu (z zadaną liczbą miejsc po przecinku).
% Przykłady:
%	\numperc{1} => 100%
%	\numperc[precision=1]{0.3541} => 35.4%
\makeatletter
%
% [NOTE] Zdefiniuj klucz, z którego będzie korzystało makro.
%	\define@key{family}{key}{...}
%	Defines a macro \KV@prefix@key with one argument. When used in a keyval list, the macro receives the value as its argument.
\define@key{numprec}{precision}{\def\mm@precision{#1}}
%
\DeclareDocumentCommand{\numperc}{O{} +m}{%
	\begingroup%
	% [NOTE] Ustaw domyślną wartość klucza `first`
	\setkeys{numprec}{precision={0},#1}%
	%  
 	\FPeval{\result}{#2 * 100}%
 	{\num[round-mode=places,round-precision=\mm@precision]{\result}\%}%
 	%
	\endgroup%
}
\makeatother

% Wyświetl liczbę z zadaną dokładnością (stałą dla całego dokumentu).
\newcommand{\dispstat}[1]{\num[round-mode=places,round-precision=2]{#1}}

%------------------------------------------------------------------------------
% Debuggowanie.
% SEC: debugging
%------------------------------------------------------------------------------

% Facilitate the conditional compilation
\usepackage{etoolbox}  % umożliwia kompilację warunkową [\newtoggle, \toggletrue, \togglefalse, \iftoggle, ...]

\newtoggle{GFXDEBUG}  % zdefiniuj flagę `GFXDEBUG`
%\togglefalse{GFXDEBUG}
\toggletrue{GFXDEBUG}  % nadaj fladze `GFXDEBUG` wartość `true`

% Wyświetlenie grafiki z opcją `draft` spowoduje wyświetlenie ramki zamiast faktycznego obrazka - co przyspiesza "robocze" generowanie dokumentu:
%   \includegraphics[draft]{...}
% Po zdefiniowaniu polecenia, którego wartość wynosi "draft" w przypadku generowania "roboczego" i "" w przypadku generowania "normalnego" i umieszczeniu go w każdym poleceniu `\includegraphics`, możemy w efektywny sposób zmieniać sposób generowania dla całego dokumentu:
%   \includegraphics[\draftgraphics]{...}
\iftoggle{GFXDEBUG}{
	% Debug ON
	\newcommand{\draftgraphics}{draft}
}{
	% Debug OFF
	\newcommand{\draftgraphics}{}
}

%------------------------------------------------------------------------------
% Konfiguracja listingów (fragmentów kodu).
% SEC: listings
%------------------------------------------------------------------------------

\usepackage{listings}

\definecolor{mygreen}{RGB}{28,172,0}
\definecolor{mylilas}{RGB}{170,55,241}
\definecolor{mycolor1}{rgb}{0.000,0.502,0.502}
\definecolor{mycolor2}{rgb}{0.502,0.000,0.502}
\definecolor{Green}{rgb}{0.1,0.5,0.1}
\definecolor{MyDarkGreen}{rgb}{0.0,0.4,0.0}

\lstloadlanguages{TeX}%

% #include-s etc.
%	morecomment=[l][\color{magenta}]{\#}

%\lstset{%
\lstdefinestyle{mytex}{%
	language=TeX,
	frame=single,                           %  Single frame around code
	basicstyle=\small\ttfamily,             %  Use small true type font
	keywordstyle=[1]\color{Blue}\bfseries,  %  MATLAB functions bold and blue
	keywordstyle=[2]\color{Purple},         %  MATLAB function arguments purple
	keywordstyle=[3]\color{Blue}\underbar,  % User functions underlined and blue
	identifierstyle=,                       % Nothing special about identifiers
	% Comments small dark green courier
	commentstyle=\usefont{T1}{pcr}{m}{sl}\color{MyDarkGreen}\small,
	stringstyle=\color{Purple},             % Strings are purple
	showstringspaces=false,                 % Don't put marks in string spaces
	tabsize=2,
	%
	%%% Put standard language functions not included in the default
	%%% language here
	morekeywords={override},
	%
	%%% Put language function parameters here
	morekeywords=[2]{},
	%
	%%% Put user defined functions here
	morekeywords=[3]{},
	%
	deletekeywords={},
	%
	morecomment=[l][\color{Blue}]{...},     % Line continuation (...) like blue comment
	numbers=none,
	%        numbers=left,                           % Line numbers on left
	firstnumber=1,                          % Line numbers start with line 1
	numberstyle=\tiny\color{Blue},          % Line numbers are blue
	stepnumber=5,                           % Line numbers go in steps of 5
	%  breaklines=true,
	literate={ą}{{\k{a}}}1
           {ć}{{\'c}}1
           {ę}{{\k{e}}}1
           {ó}{{\'o}}1
           {ń}{{\'n}}1
           {ł}{{\l{}}}1
           {ś}{{\'s}}1
           {ź}{{\'z}}1
           {ż}{{\.z}}1
           {Ą}{{\k{A}}}1
           {Ć}{{\'C}}1
           {Ę}{{\k{E}}}1
           {Ó}{{\'O}}1
           {Ń}{{\'N}}1
           {Ł}{{\L{}}}1
           {Ś}{{\'S}}1
           {Ź}{{\'Z}}1
           {Ż}{{\.Z}}1,	
}

\lstset{style=mytex}

% Spraw, aby asterysk w czcionkach o stałej szerokości (np. Courier) był wyświetlany na wyższym poziomie.
\makeatletter
\lst@CCPutMacro
\lst@ProcessOther {"2A}{%
	\lst@ttfamily 
	{\raisebox{2pt}{*}}% used with ttfamily
	\textasteriskcentered}% used with other fonts
\@empty\z@\@empty
\makeatother

%------------------------------------------------------------------------------
% Varia.
% SEC: VARIA
%------------------------------------------------------------------------------

\usepackage{authblk}  % umożliwia podanie afiliacji autorów, gdy różni z różnych jednostek naukowych [\author, \affil]

\newcommand{\Lab}{\mbox{CIE $\textnormal{L}^{*}\textnormal{a}^{*}\textnormal{b}^{*}$}}

% Maksymalna "głębokość" numeracji.
\setcounter{secnumdepth}{4}

% Polecenie umożliwiające wstawienie "komentarza" w tekście (który nie będzie wyświetlany).
\newcommand{\ignore}[1]{}

\usepackage{todonotes}  % m.in. znacznik `\todo`

% Ręcznie zdefiniowany znacznik TODO.
%\newcommand{\todo}[1]{\noindent\colorbox{red}{\textcolor{white}{\bf\textsf{TODO}:}}~\textcolor{red}{#1}}


%------------------------------------------------------------------------------
% Definicje własnych wykresów.
% SEC: wykresy
%------------------------------------------------------------------------------

% [NOTE] Utwórz polecenie `\PlotTDHist` (do generowania histogramów 2D) o trzech parametrach:
%	#1 - skala wykresu (domyślnie 1.0)
%	#2 - tytuł wykresu
%	#3 - nazwa pliku z danymi wykresu
\newcommand{\PlotTDHist}[3][1.0]%
{%
	\pgfmathsetmacro{\plotscale}{#1}%
	\newcommand{\plottitle}{#2}%
	\newcommand{\filename}{#3}%
	% [NOTE]: Polecenie `\input` służy do wklejania zawartości jednego pliku do drugiego (działa podobnie do #include w C/C++).
	\definecolor{mycolor1}{rgb}{0.00000,0.44700,0.74100}%
%
\begin{tikzpicture}
\pgftransformscale{\plotscale}
\begin{axis}[%
width=\imgwidth,
height=\imgheight,
scale only axis,
xmin=40,
xmax=220,
xtick={ 50,  70,  90, 110, 130, 150, 170, 190, 210},
xlabel style={font=\color{white!15!black}},
xlabel={width [px]},
ymin=40,
ymax=420,
ytick={ 50,  90, 130, 170, 210, 250, 290, 330, 370, 410},
ylabel style={font=\color{white!15!black}},
ylabel={length [px]},
axis background/.style={fill=white},
title style={font=\bfseries},
title={\plottitle},
axis x line*=bottom,
axis y line*=left,
legend style={legend cell align=left, align=left, draw=white!15!black}
]
\addplot[scatter, only marks, mark=*, color=mycolor1, mark options={}, scatter/use mapped color={mark options={}, draw=mycolor1, fill=mycolor1}, visualization depends on={\thisrow{size} \as \perpointmarksize}, scatter/@pre marker code/.append style={/tikz/mark size=\perpointmarksize}] table[x=x, y=y, row sep=newline, col sep=comma]{\filename};
\end{axis}	
\end{tikzpicture}
%
}

% [NOTE] Utwórz polecenie `\PlotROC` (do generowania wykresów ROC).
\newcommand{\PlotROC}[2][1.0]%
{%
	\pgfmathsetmacro{\plotscale}{#1}%
	\newcommand{\filename}{#2}%
	\begin{tikzpicture}
\pgftransformscale{\plotscale}
\begin{axis}[
xmin=0,   xmax=1,
ymin=0,   ymax=1,
grid=major,
xtick style={draw=none},
ytick style={draw=none},
xlabel={$1 - \mathrm{TNR}$},
ylabel={$\mathrm{TPR}$},
]
\addplot[mark=none, red, ultra thick] table[x=X, y=Y, col sep=comma] {\filename};
\end{axis}
\end{tikzpicture}
%
}

% =============================================================================

% [NOTE] Te polecenia nie wyświetlają tekstu, tylko służą konfiguracji szablonów wyświetlania w przypadku generowanych stron tytułowych.
\author[1]{Pawel Kleczek}
\author[2]{Paulus Kletschek}
\affil[1]{AGH University of Science and Technology, Department of Automatic Control and Robotics, Krakow, Poland}
\affil[2]{Dummy University, Department of Dumminess, Nirgendwo, Germania}

% =============================================================================

\begin{document}

% ======================================
%     Nagłówek z informacją o autorze
%         (NIE do manuskryptu)
% ======================================

\null\hfill \textbf{ostatnia modyfikacja: 2018-12-20}

\noindent
Autor: \textbf{Pawe\l{} K\l{}eczek} (\href{mailto:pkleczek@agh.edu.pl}{\texttt{pkleczek@agh.edu.pl}})

\vspace{0.1cm}

\noindent\rule{\linewidth}{0.4pt}

\vspace{0.5cm}

% ======================================

\noindent
{\Large\bfseries A sample manuscript}

\noindent
% [NOTE] NIE stosować polskich znaków w imionach i nazwiskach autorów! To powoduje późniejsze problemy z indeksowaniem takich artykułów w bazach typu WoS...
Pawel Kleczek$^{1}$
and Paulus Kletschek$^{2}$

\begin{itemize}
	\item[] $^{1}$ \quad AGH University of Science and Technology, Department of Automatic Control and Robotics, Krakow, Poland
	\item[] $^{2}$ \quad Dummy University, Department of Dumminess, Nirgendwo, Germania
\end{itemize}


\noindent
Correspondence: \texttt{pkleczek@agh.edu.pl}; Tel.: +48 12 617 50 65

\vspace{0.5cm}

\noindent
\textbf{Abstract}

\par\medskip

\textit{Background:}
Lorem ipsum dolor sit amet.

\textit{Objectives and methods:}
Lorem ipsum dolor sit amet

\textit{Results:}
Lorem ipsum dolor sit amet

\textit{Conclusion:} 
Lorem ipsum dolor sit amet

\vspace{0.3cm}

\noindent
\textbf{Keywords:}
keyword1, keyword2, keyword3


% ======================================

\section{Introduction}
% [NOTE] Umieszczenie polecenia `\label` umożliwia późniejsze odwołanie się w tekście do tej sekcji (odpowiedni numer sekcji w odnośniku zostanie ustawiony automatycznie).
\label{sec:introduction}

Stworzyłem niniejszy dokument po to, aby ułatwić przygotowywanie manuskryptów publikacji do czasopism anglojęzycznych akceptujących dokumenty złożone z użyciem systemu \LaTeX (albo wręcz wymagających użycia tego systemu).
%
Zwróć uwagę na to, że wiele czasopism dostarcza autorowi swoje własne szablony w postaci paczek albo klas, których użycie jest wmagane podczas składu manuskryptu. Niektóre z prezentowanych w tym dokumencie rozwiązań może być niekompatybilnych z takimi szablonami (choć w takich przypadkach starałem się podać w kodzie również rozwiązanie mniej wygodne, ale przynajmniej działające -- korzystające wyłącznie ze standardowych pakietów).

Moim celem nie jest przedstawianie ,,podstaw'' korzystania z \LaTeX-a oraz m.in. takich zagadnień, jak składanie równań czy tworzenie odpowiedniej struktury dokumentu. Postanowiłem skupić się na aspekcie \textit{efektywnego} tworzenia \textit{manuskryptu} -- w takim przypadku ważne jest m.in. zachowanie spójności danych (np. skuteczności metody podawanej w abstrakcie, w sekcji poświęconej omówieniu wyników i w podsumowaniu), możliwość szybkiego generowania nowych wykresów po wprowadzeniu ulepszeń do metody itd.

Jeśli w którymś miejscu tłumaczę rzeczy, które można uznać za ,,podstawy'' (np. użycie twardej spacji), to robię to tylko dlatego, że jestem purystą językowym i boli mnie, że wiele osób nie przestrzega podstawowych zasad dotyczących interpunkcji i składu tekstu -- co często wynika nie z ich złej woli, a zwykłej niewiedzy. Pragnę więc ,,nieść kaganek oświaty'', co (mam nadzieję) będzie z pożytkiem dla nas wszystkich\textellipsis ;)

\par\bigskip

Jeśli stwierdzisz, że moja praca Ci się przydała, lub jeśli masz konstruktywne uwagi odnośnie tego, co poprawić -- daj mi proszę znać, pisząc na adres \href{mailto:pkleczek@agh.edu.pl}{\texttt{pkleczek@agh.edu.pl}}.


\section{Material and methods}

\subsection{Preambuła}
\label{sub:preamble}

Zwróć uwagę, że kolejność pakietów dołączanych w preambule \textbf{ma znaczenie}!
Czasem zła kolejność prowadzi do błędów kompilacji, gdyż np. pakiet~A redefiniuje standardowe polecenie sposób uniemożliwiający korzystanie w pakiecie~B w sposób zamierzony przez twórców pakietu~B -- wówczas należy zamienić miejscami polecenia \lstinline|\usepackage| dołączające te pakiety.


\subsection{Odnośniki}
\label{sub:references}

Aby odwołać się do sekcji, ilustracji, tabeli lub pozycji w wyliczeniu użyj polecenia \lstinline|\ref|, np.:
sekcja~\ref{sec:introduction}, rys.~\ref{fig:plot_generation_custom_command_example}, tab.~\ref{tab:threeparttable}, poz.~\ref{itm:enumitem2}.
Oczywiście wcześniej należy umieścić w/przy odpowiednim elemencie etykietę za pomocą polecenia \lstinline|\label|.

\par\bigskip

W przypadku umieszczania wielu odnośników do pozycji bibliograficznych w ramach elementu \lstinline|\cite|, np. \cite{bibitem1,bibitem2}, \textbf{nie należy} rozdzielać identyfikatorów tych pozycji spacjami -- w przypadku niektórych systemów obsługi bibliografii umieszczanie takich spacji prowadzi to do błędów kompilacji.\\
Poprawny sposób zapisu: \lstinline|\cite{bibitem1,bibitem2}| (a \textbf{nie} \lstinline|\cite{bibitem1, bibitem2}|).

\par\bigskip

% SEC: \eqref

Aby odwołać się do ponumerowanego równania:
\begin{equation}\label{eq:equation1}
a^2 + b^2 = c^2
\end{equation}
tak, aby numer był otoczony nawiasami, należy skorzystać z polecenia \lstinline|\eqref|, przykładowo: równanie~\eqref{eq:equation1}.

\subsection{Wyliczenia}
\label{sub:enumerations}

Aby utworzyć wyliczenie w tekście akapitu, korzystamy ze środowiska \verb|enumerate*| z pakietu \verb|enumitem|:
\begin{enumerate*}[(1)]
	% [NOTE] Aby móc odwołać się do danej pozycji wyliczenia, dodaj `\label{ą}`.
	\item\label{itm:enumitem1} pozycja~1, oraz
	\item\label{itm:enumitem2} pozycja~2.
\end{enumerate*}
Możemy odwoływać się do poszczególnych pozycji wyliczenia za pomocą \lstinline|\ref|, np.: \ref{itm:enumitem1}.

\par\smallskip

W przypadku, jeśli wydawca narzuca użycie pakietu \verb|enumerate|, utworzyć wyliczenie w tekście akapitu musimy użyć środowiska \lstinline|inparaenum|~(zob. przykład~\ref{lst:inparaenum}):
\begin{lstlisting}[label={lst:inparaenum}]
\usepackage{enumerate}

\begin{inparaenum}[(1)]
	\item pozycja 1,
	\item pozycja 2.
\end{inparaenum}
\end{lstlisting}
O zaletach \verb|enumitem| nad \verb|enumerate| przeczytasz \href{https://tex.stackexchange.com/a/222412/44391}{tutaj}.

\par\smallskip

W przypadku, jeśli wydawca narzuca użycie pakietu \lstinline|enumerate|, aby zmienić rodzaj wyliczenia -- np. na małe litery w nawiasie -- musimy podać parametr opcjonalny \lstinline|label|, np. \lstinline|label=(\alph*)|:
\begin{enumerate}[label=(\alph*)]
	\item pozycja~1
\end{enumerate}


\subsection{Skład tekstu}
\label{sub:typesetting}

% SEC: ~

Pamiętaj, aby (w zasadzie) \textbf{zawsze} stosować twardą spację (w \LaTeX-u: ,,\verb|~|'') m.in. przy podawaniu odnośników do ilustracji i tabel (za pomocą \lstinline|\ref|) oraz do bibliografii (za pomocą \lstinline|\cite|) -- takie są zasady składu (a dodatkowo w ten sposób unikamy ,,przerzucenia'' numeru do nowej linii), np.: \lstinline|Rys.~\ref{fig:fig1}|.

% SEC: \mbox
W niektórych przypadkach twarde spacje nie działają (np. wokół dywizów i półpauz), dlatego w takich przypadkach należy skorzystać z polecenia \lstinline|\mbox|, np. \lstinline|\mbox{$r$ -- the radius}|.

\par\smallskip

Z polecenia \lstinline|\mbox| korzystamy również wówczas, gdy chcemy zapobiec podziałowi równania umieszczonego w tekście -- porównaj poniższe akapity:
\begin{enumerate}
	\item Oto krótki akapit zawierający naprawdę dużą ilość mądrego tekstu oraz pewne równanie $a + b + c + d = 100$, które nie powinno być dzielone pomiędzy wierszami.
	\item Oto krótki akapit zawierający naprawdę dużą ilość mądrego tekstu oraz pewne równanie \mbox{$a + b + c + d = 100$}, które nie powinno być dzielone pomiędzy wierszami -- tym razem zastosowaliśmy polecenie \lstinline|\mbox|. \textbf{Uwaga:} Należy wówczas tak zredagować tekst, aby równanie nie wchodziło na margines!
\end{enumerate}

\par\bigskip

% SEC: --
Pamiętaj, o różnicy między \textbf{pauzą} (in. \textbf{myślnikiem}, ,,---''), \textbf{półpauzą} (,,--'') oraz \textbf{dywizem} (,,-''). Do oznaczania wtrąceń służą półpauzy -- w kodzie \LaTeX-a należy użyć ,,\verb|--|'' (dwa minusy) -- a \textbf{nie} ,,\verb|-|'' (jeden minus)!

\par\smallskip

Podobnie zakresy wartości podajemy zawsze z użyciem półpauzy (np. 1--3), a \textbf{nie} z użyciem dywizu (1-3)!

\par\bigskip

% SEC: ``...''
W języku angielskim cudzysłowy tzw. drukarskie zapisujemy w formie ``\textellipsis'' (a \textbf{nie}  ,,polskich'' lub tzw. "prostych") -- w tym celu należy pisać \verb|``abc''| (dwa symbole ,,\verb|`|'' oraz dwa symbole apostrofu ,,\verb|'|'').

\par\bigskip

% SEC: odstępy przed jednostkami
Symbolu procentu nie poprzedzamy spacją: 1\% (a \textbf{nie} 10~\%). Pozostałe wyrażenia ,,wartość + symbol'' składamy za pomocą pakietu \hyperref[sub:siunitx]{\texttt{siunitx}}.

\par\bigskip

% SEC: ith

W przypadku odwołania do ,,$i$-tego'' elementu w języku angielskim zwykło się pisać ,,$i$th'' \textbf{bez} dywizu (w kodzie \LaTeX: \lstinline|$i$th|).


\subsection{Ilustracje}
\label{sub:figures}


Rysunek~\ref{fig:plot_generation_custom_command_example} pokazuje użycie zdefiniowanego przez użytkownika polecenia do generowania wykresu.

\begin{figure}[H]
	\centering

	\newcommand{\plotscale}{0.6}
	\PlotROC[\plotscale]{data/roc_plot.csv}
	
	\caption{\label{fig:plot_generation_custom_command_example}Przykład użycia zdefiniowanego przez nas polecenia do generowania wykresu.}
\end{figure}

Aby odwołać się w opisie grupy ilustracji do konkretnej ilustracji, użyj \lstinline|\subref{fig:xxx}|~(Rys.~\ref{fig:subrefs}). \textbf{Uwaga:} Polecenie \lstinline|\subref| wymaga poprzedzenia przez \lstinline|\protect| albo skorzstania z \lstinline|\cprotect\caption|!
Przykład ten dodatkowo pokazuje użycie zdefiniowanych przez użytkownika długości do zachowania spójności układu grupy ilustracji.

\begin{figure}[H]
	\setlength{\picwidth}{0.3\groupwidth}
	\setlength{\subfigshsep}{0.05\linewidth}

	\centering

	\begin{subfigure}[c]{\picwidth}
		\centering
		\includegraphics[\draftgraphics,width=\textwidth]{images/sample1.jpg}
		% [NOTE] Polecenie \caption* tworzy podpis bez numeru (np. "(a)").
		\caption*{Brak ,,numeru'' ilustracji}
	\end{subfigure}

	\vspace{\subfigsvsepdef}

	\begin{subfigure}[c]{\picwidth}
		\centering
		\includegraphics[\draftgraphics,width=\textwidth]{images/sample1.jpg}
		\caption{Podpis pod ilustracją}
		\label{fig:sampleX_sub1}
	\end{subfigure}
	%
	\hspace{\subfigshsep}
	%
	\begin{subfigure}[c]{\picwidth}
		\centering
		\includegraphics[\draftgraphics,width=\textwidth]{images/sample2.jpg}
		\caption{Abc}
		\label{fig:sampleX_sub2}
	\end{subfigure}
	
	\cprotect\caption{\label{fig:subrefs} Odwołanie się w opisie do konkretnej ilustracji w grupie: (\subref{fig:sampleX_sub1}) oraz (\protect\subref{fig:sampleX_sub2}).}
\end{figure}

% ----

Opis musi się zaczynać zaraz za nawiasem klamrowym \verb|}| od \lstinline|\label| -- inaczej zostanie wyświetlona spacja, co widać szczególnie wyraźnie w przypadku długiego podpisu (por. Rys.~\ref{fig:long_caption_with_space}~i~\ref{fig:long_caption_without_space}.

\begin{figure}[H]
	\centering
	\includegraphics[\draftgraphics,height=2cm]{images/dummy.png}
	\cprotect\caption{\label{fig:long_caption_with_space} W opisie tej ilustracji dodaliśmy spację między nawiasem klamrowym od \verb|\label|, przez co w przypadku długiego opisu widać, że ,,coś jest nie tak'' z odstępem po numerze ilustracji.}
\end{figure}

\begin{figure}[H]
	\centering
	\includegraphics[\draftgraphics,height=2cm]{images/dummy.png}
	\cprotect\caption{\label{fig:long_caption_without_space}W opisie tej ilustracji \textbf{nie} dodaliśmy spacji między nawiasem klamrowym od \verb|\label|, przez co nawet w przypadku długiego opisu odstępy są prawidłowe.}
\end{figure}

% ----

\begin{figure}[H]
	\setlength{\fboxsep}{0pt}  % określa odległość ramki od obrazu
	\setlength{\fboxrule}{2pt}  % określa grubość ramki
	
	\centering

	\fbox{\includegraphics[\draftgraphics,height=1cm]{images/img1.jpg}}
	
	% [NOTE] Długość `\captionmargin` określa rozmiar marginesu po każdej ze stron podpisu.
	\setlength{\captionmargin}{2cm}
	\cprotect\caption{\label{fig:fbox}Tworzenie ramki wokół grafiki za pomocą polecenia \verb|\fbox|.}
\end{figure}

\begin{figure}[H]
	\centering
	\includegraphics[\draftgraphics,height=3cm]{images/border_normals.png}
	\cprotect\caption{\label{fig:border_normals}Stosowanie pewnych poleceń środowiska matematycznego w podpisie wymaga użycia \lstinline|\cprotect| przed \lstinline|\caption| albo poprzedzania każdego takiego ,,trefnego'' polecenia matematycznego poleceniem \lstinline|\protect|, np.: 
	$\overrightarrow{P_i}$, $f_\protect\mathrm{x}$.}
\end{figure}

% ----

Rysunki \ref{fig:minipage_figl} i \ref{fig:minipage_figr} (dwa w jednym rzędzie) zostały wykonane z użyciem środowiska \lstinline|figure*| w połączeniu ze środowiskami \lstinline|minipage| -- nie ma możliwości umieszczenia wielu środowisk \lstinline|figure| w jednym rzędzie.

\begin{figure*}
	\newlength{\figlwidth}  % szerokość lewej ilustracji
	\newlength{\figrwidth}  % szerokość prawej ilustracji
	\setlength{\figlwidth}{.4\linewidth}
	\setlength{\figrwidth}{\linewidth - \figlwidth}  % cała grupa zajmuje pełną szerokość linii

	\setlength{\imgheight}{3cm}
	
	% [NOTE] Musimy zmniejszyć margines na tyle, żeby podpisy się nie "rozjeżdżały".
	\setlength{\captionmargin}{0.5cm}
	
	\centering
	\begin{minipage}[b]{\figlwidth}
		\centering
		% [TODO-PK] ew. pokazać z użyciem "natwidth=NW, natheight=NH"
		\includegraphics[\draftgraphics,height=\imgheight,width=\linewidth]{images/img1.png}
	\end{minipage}%
	\hspace*{0.2cm}
	\begin{minipage}[b]{\figrwidth}
		\centering
		\includegraphics[\draftgraphics,height=\imgheight,width=\linewidth]{images/img2.png}
	\end{minipage}\\[-4pt]  % [NOTE] Konieczne ręczne wyrównanie odległości między grafikami a podpisem, aby była ona spójna ze środowiskiem `figure`.
	\begin{minipage}[t]{\figlwidth}
		\centering
		\caption{Lewa ilustracja.}
		\label{fig:minipage_figl}
	\end{minipage}%
	\hspace*{0.2cm}
	\begin{minipage}[t]{\figrwidth}
		\centering
		\cprotect\caption{Prawa ilustracja zawiera dłuższy podpis. Należy ustawić wartość długości \lstinline|\captionmargin| tak, aby się on nie rozjeżdżał.}
		\label{fig:minipage_figr}
	\end{minipage}%
	\hspace*{0.3cm}
\end{figure*}

\subsection{Tabele}
\label{sub:tables}

Proste, eleganckie tabele tworzymy za pomocą środowiska \texttt{tabularx}, np. Tab.~\ref{tab:simple_tabularx}. Środowisko to pozwala m.in. zdefiniować \textit{bezwzględną} szerokość całej tabeli oraz \textit{względną} szerokość poszczególnych kolumn.

\begin{table}[H]
	\centering
	
	\caption{Prosta tabela utworzona za pomocą środowiska \texttt{tabularx}.}
	\label{tab:simple_tabularx}
	
	% [NOTE] Zmień wysokość wiersza, aby uzyskać więcej światła (i tym samym poprawić czytelność tabeli).
	\renewcommand{\arraystretch}{1.3}
	
	% [NOTE] Zmień odstęp między kolumnami, aby uzyskać więcej światła (i tym samym poprawić czytelność tabeli).
	\setlength{\tabcolsep}{4mm}	
	
	% [NOTE] Cała tabela zajmuje 90% szerokości wiersza, a (trzy) kolumny kolejno: 0.8/N, 1.6/N oraz 0.6/N szerokości tabeli, gdzie N to liczba kolumn. UWAGA: Wartości podane w parametrach `\hsize` muszą się sumować do N!
	\begin{tabularx}{0.9\textwidth}{>{\hsize=0.8\hsize}X >{\hsize=1.6\hsize}X >{\hsize=0.6\hsize}X}
		\toprule
		\textbf{A} & \textbf{B} & \textbf{C} \\
		\midrule
		a & b & c \\
		\midrule
		d & e & f \\
		\bottomrule
	\end{tabularx}
	
\end{table}

\par\bigskip

Czasem potrzebujemy umieścić na dole tabeli przypisy -- wówczas ,,opakowujemy'' środowisko \texttt{tabularx} w środowisko \lstinline|threeparttable| (zob.~Tab.~\ref{tab:threeparttable}). Przypisy w ,,treści'' tabeli dodawane są za pomocą polecenia \lstinline|\tnote|, natomiast ich opis umieszcza się w środowisku \lstinline|tablenotes|.

\begin{table}[H]
	\centering
	
	\begin{threeparttable}
		\cprotect\caption{Tabela utworzona za pomocą środowiska \lstinline|threeparttable|.}
		\label{tab:threeparttable}
		
		% [NOTE] Ustaw odpowiednie formatowanie nagłówków tabeli utworzonych za pomocą polecenia `\thead`.
		\renewcommand\theadfont{\bfseries}
		
		\begin{tabularx}{0.9\linewidth}{>{\hsize=.8\hsize}X>{\hsize=1.4\hsize}X>{\hsize=.8\hsize}X}
			\toprule
			% [NOTE] Utwórz spójne nagłówki tabeli z pomocą pakietu `makecell`.
			\thead{A} & \thead{B} & \thead{C} \\
			\midrule
			x & y & XXX\tnote{a} \\
			& qrst & XXX \\
			\bottomrule
		\end{tabularx}
		
		\begin{tablenotes}
		   	\footnotesize
		   	\item[a] przypis w tabeli
		\end{tablenotes}
	 		
	\end{threeparttable}
\end{table}

Tabela~\ref{tab:table_with_graphics} zawiera kolumnę typu ,,akapit teksu'' (\lstinline|m{...}|) oraz grafiki. Zwróć uwagę, że w przypadku tabel zawierających kolumny takiego typu nie możemy stosować domyślnego sybolu końca wiersza ,,\lstinline|\\|'' (bo byłby konflikt ze znakiem nowego wiersza w akapicie) -- zamiast tego stosujemy polecenie \lstinline|\tabularnewline|.

\begin{table}[H]
	\setlength{\imgheight}{1cm}
	
	\centering
	
	\caption{\label{tab:table_with_graphics}Tabela zawierająca grafiki, utworzona za pomocą środowiska \lstinline|tabular|.}
	
	% [FIXME-PK]: Jak wyrównać kolumny z liczbami względem kropki dziesiętnej (do prawej strony?) w takiej złożonej tabeli?
	% https://tex.stackexchange.com/questions/47422/siunitx-how-to-deal-with-invalid-numerical-input
	\begin{tabular}{m{3cm} ccc}
		\toprule
		% [NOTE] Aby ilustracje nie rozjeżdżały się w przypadku szerszego wiersza, należy je wyrównać w pionie do środka - za pomocą opcjonalnego argumentu `align=c` z pakietu `graphbox`.
		Original image &
		 \includegraphics[\draftgraphics,height=\imgheight, align=c]{images/A.png}  &
		 \includegraphics[\draftgraphics,height=\imgheight, align=c]{images/B.png}  &
		 \includegraphics[\draftgraphics,height=\imgheight, align=c]{images/C.png}
		\tabularnewline
		%
		\midrule
		%
		% [NOTE] W przypadku tabel zawierających kolumny typu `m{}` nie możemy stosować domyślnego sybolu końca wiersza `\\` (bo byłby konflikt ze znakiem nowego wiersza w akapicie) - zamiast tego stosujemy polecenie `\tabularnewline`.
		Area1 & 0.10 & -0.09 & -0.13 \tabularnewline
		Area2 & 0.1 & 0.02 & 10 \tabularnewline
		\bottomrule
	\end{tabular}
	
\end{table}

\par\bigskip

Tabela~\ref{tab:multi_table} zawiera komórki scalone ze sobą w wierszach i w kolumnach. Polecenie \lstinline|\cline| służy do rysowania poziomej krawędzi tylko w zadanych kolumnach.

\begin{table}[h]
	\centering
	\caption{\label{tab:multi_table}Macierz pomyłek}
	
	\renewcommand{\arraystretch}{1.3}
	
	\begin{tabularx}{0.9\textwidth}{C{1} C{0.4} C{1.8} C{1.8}}
		\toprule
		& & \multicolumn{2}{ C{3.6} }{obserwowane klasy rzeczywiste} \\
		\cmidrule(lr){3-4}
		& & \raisebox{0.3em}{$\boldsymbol{\oplus}$} & \raisebox{0.3em}{$\boldsymbol{\ominus}$} \\ \cline{1-4}
		\multicolumn{1}{ C{1}  }{\multirow{2}{\linewidth}[-6pt]{przewidywane\\klasy\\decyzyjne} } &
		\multicolumn{1}{ Y }{$\boldsymbol{\oplus}$} & Prawdziwie Pozytywne \newline (TP -- \textit{True Positives}) & Fałszywie Pozytywne \newline (FP -- \textit{False Positives}) \\
		\multicolumn{1}{ Y }{} &
		\multicolumn{1}{ Y }{$\boldsymbol{\ominus}$} & Fałszywie Negatywne \newline (FN -- \textit{False Negatives}) & Prawdziwie Negatywne \newline (TN -- \textit{True Negatives}) \\
		\bottomrule
	\end{tabularx}
\end{table}

\par\bigskip

Dodatkowe wskazówki odnośnie tworzenia eleganckich tabeli: \href{https://www.inf.ethz.ch/personal/markusp/teaching/guides/guide-tables.pdf}{Small Guide to Making Nice Tables (Markus Püschel)}


\subsection{Pakiet \lstinline|siunitx|}
\label{sub:siunitx}

Pakiet \lstinline|siunitx| to potężne narzędzie umożliwiające efektywny skład tekstów zawierających wielkości -- niekoniecznie fizyczne. Ogólnie chodzi o wszelkie wyrażenia w postaci ,,wartość + jednostka''.

\par\bigskip

% SEC: \SI, \SIrange; tekst vs. półpauza

Pojedyncze wielkości wyświetlaj za pomocą polecenia \lstinline|\SI|, np. \SI{5}{\micro\meter}.
%
Zakresy wielkości wyświetlaj za pomocą polecenia \lstinline|\SIrange|, np. \SIrange{1}{3}{\micro\meter}.
%
Aby zamiast półpauzy użyć słowa (np. ,,\textit{to}''), skorzystaj z opcji \lstinline|range-phrase={ to }|, np. \SIrange[range-phrase={ to }]{1}{3}{\micro\meter}.

\par\bigskip

% SEC: zaokrąglanie

Za pomocą opcji \lstinline|round-mode=places| w połączeniu z \lstinline|round-precision=N| możesz kontrolować wyświetlaną liczbę miejsc po przecinku, np. dla \lstinline|\num[round-mode=places,round-precision=2]{0.009}|: \num[round-mode=places,round-precision=2]{0.009}.

\par\bigskip

% SEC: per-mode=symbol

Domyślnie ,,na'' (np. ,,radiany \textit{na} sekundę'') wyświetlane jest jako potęga o ujemnym wykładniku (\SI{0.1}{\radian\per\second}).
Aby ,,na'' było wyświetlany jako symbol ,,/'', należy użyć argumentu \lstinline|per-mode=symbol|: \SI[per-mode=symbol]{0.1}{\radian\per\second}.

\par\bigskip

% SEC: "ręczna" definicja jednostek

Jeśli potrzebujesz skorzystać z niestandardowej jednostki (np. piksel), możesz ją zdefiniować ręcznie za pomocą polecenia \lstinline|\DeclareSIUnit|, np. \lstinline|\DeclareSIUnit\pixel{px}|.

\par\bigskip


% SEC: \num (zmiana czcionki!)

Zwróć uwagę, że cyfry wyświetlane z użyciem \lstinline|\num| mają nieco inny krój, niż cyfry w zwykłym tekście: zwykły tekst -- 123, \lstinline|\num| -- \num{123}.


\cprotect\subsection{Pakiet \lstinline|fp|}

Pakiet \lstinline|fp| pozwala m.in. na definiowanie stałych zmiennoprzecinkowych oraz na wykonywanie na nich operacji arytmetycznych -- dzięki czemu znakomicie nadaje się do parametryzacji manuskryptu (np. liczba próbek, wartości skuteczności algorytmu, parametry wykresów itd.).

\par\bigskip

Do definiowania stałych służy polecenie \lstinline|\FPeval|, np. \lstinline|\FPeval{\inta}{(1)}|.

% [NOTE] Wartości w poleceniach `\FPeval` należy umieszczać w nawiasach okrągłych, m.in. po to, aby poprawnie działały wartości ujemne (inaczej w pewnych przypadkach występują błędy parsowania - zob. niżej).
\FPeval{\ncorrect}{(6)}
\FPeval{\nincorrect}{(10)}

\FPeval{\myfpfloatpositive}{(0.207)}
\FPeval{\myfpfloatnegative}{(-0.500)}

% [NOTE] Brak nawiasów spowoduje błędy kompilacji ("FP error: UPN stack is empty!").
% \FPeval{\myfpfloatnegativenoparentheses}{-0.300}

% [NOTE] Polecenie `\FPeval` umożliwia wykonywanie prostych operacji arytmetycznych.
\FPeval{\accuracy}{\ncorrect / (\ncorrect + \nincorrect)}

\par\bigskip

Warto zdefiniować polecenie, z którego będziemy korzystać w całym dokumencie dla wyświetlania ,,standardowych'' wyników ułamkowych z zadaną dokładnością (jednolitą dla całego dokumentu), \linebreak np. \lstinline|\dispstat{\accuracy}| -- \dispstat{\accuracy}.

Niniejszy dokument definiuje polecenie \lstinline|\numperc| służące do wyświetlania ułamka jako procentu (z zadaną liczbą miejsc po przecinku), np.: \lstinline|\numperc{0.767}| -- \numperc{0.767}, \lstinline|\numperc[precision=1]{0.76732}| -- \numperc[precision=1]{0.76732}.

% ----

% SEC: polecenia \FPeval w osobnym pliku

% [NOTE] Definicje wartości (tj. polecenia `\FPeval`) wygodnie jest umieścić w osobnym pliku, generowanym np. z poziomu MATLAB-a (wtedy podmiana wyników polega na podmianie pojedynczego pliku).
\FPeval{\overallaccuracy}{0.921}


% ----

% SEC: nawiasy klamrowe wokół makr fp

\FPeval{\inta}{(8)}

Odwołując w tekście się do wartości zdefiniowanych za pomocą \lstinline|\FPeval| (np. \lstinline|\FPeval{\inta}{1}|) należy umieścić je wewnątrz nawiasów klamrowych -- inaczej nie zostaną dodane spacje wokół wartości. Porównaj: wersja bez klamer -- ,,tekst \inta tekst'', wersja z klamrami -- ,,tekst {\inta} tekst''.

\subsection{Pakiet \lstinline|pgfplots|}
\label{sub:pgfplots}

Pakiet \lstinline|fp| przydaje się m.in. podczas generowania histogramów, w przypadku których pierwszy i ostatni z przedziałów klasowych zawiera wszystkie wartości odpowiednio mniejsze albo większe od wartości skrajnych (np. $(-\infty, l)$ i $(r, +\infty)$) -- taką stytuację warto bowiem opisać w podpisie histogramu, a dzięki użyciu poleceń \lstinline|\FPeval| zachowujemy spójność między kodem użytym do wygenerowania histogramu, a opisem w kodzie \LaTeX.
%
Jeśli prostokąty na histogramie wyglądają ,,dziwnie'' (np. zlewają się ze sobą), należy dobrać odpowiednią wartość parametru \lstinline|bar width| w odpowiednim elemencie środowiska \lstinline|tikzpicture|.

\begin{figure}[H]
	\setlength{\imgwidth}{10cm}
	\setlength{\imgheight}{3cm}
	
	% [NOTE] Wczytaj wartości parametrów konfiguracyjnych histogramu z pliku.
	\FPeval{\xmin}{(0.200)}  % lewy kraniec przedzialu wartosci
\FPeval{\xmax}{(0.600)}  % prawy kraniec przedzialu wartosci
\FPeval{\binwidth}{(0.100)}  % szerokosc przedzialu klasowego

\FPeval{\ymaxaxis}{(25)}  % maksymalna wartosc wyswietlana na osi Y


	\centering
	% This file was created by matlab2tikz.
%
%The latest updates can be retrieved from
%  http://www.mathworks.com/matlabcentral/fileexchange/22022-matlab2tikz-matlab2tikz
%where you can also make suggestions and rate matlab2tikz.
%
\definecolor{mycolor1}{rgb}{0.00000,0.44700,0.74100}%
%

\begin{tikzpicture}

\pgfmathsetmacro{\xminpt}{\xmin - \binwidth / 2}
\pgfmathsetmacro{\xmaxpt}{\xmax + \binwidth / 2}

\begin{axis}[%
width=\imgwidth,
height=\imgheight,
scale only axis,
bar shift auto,
bar width=\binwidth,
xmin=\xminpt,
xmax=\xmaxpt,
%
xlabel style={font=\color{white!15!black}},
xlabel={$\delta x$},
ymin=0,
%
% [NOTE] Wartość `ymax` należy dostosować do liczby próbek w najliczniejszym przedziale klasowym.
ymax=\ymaxaxis,
%
ylabel style={font=\color{white!15!black}},
ylabel={\# of cases},
axis background/.style={fill=white},
title style={font=\bfseries},
title={Relative errors},
xtick pos=left,
]
% [NOTE] Dane odczytywane z pliku w formacie TSV, w którym wiersze oddzielone są symbolem nowej linii. Aby wczytać dane z formatu CSV, użyj argumentu `col sep=comma`.
\addplot[ybar, fill=mycolor1, draw=black, area legend] table[col sep=tab,row sep=newline]{data/histogram_simple.tsv};

\addplot[forget plot, color=white!15!black] table[row sep=crcr, col sep=comma] {%
\xminpt,0\\
\xminpt,0\\
};

\end{axis}
\end{tikzpicture}%
	% [NOTE] Wartości krańców przedziałów klasowych zostały umieszczone w pliku `histogram_simple_fp`.
	\caption{\label{fig:histogram_simple}A simple histogram (the first bin also includes all values smaller than \num[round-mode=places,round-precision=2]{\xmin} and the last bin also includes all values larger than \num[round-mode=places,round-precision=2]{\xmax}).}
\end{figure}


\begin{figure}[H]
	\setlength{\imgwidth}{15cm}
	\setlength{\imgheight}{6cm}

	\FPeval{\xmin}{(0.200)}  % lewy kraniec przedzialu wartosci
\FPeval{\xmax}{(0.600)}  % prawy kraniec przedzialu wartosci
\FPeval{\binwidth}{(0.100)}  % szerokosc przedzialu klasowego

\FPeval{\ymaxaxis}{(40)}  % maksymalna wartosc wyswietlana na osi Y


	\centering
	% This file was created by matlab2tikz.
%
%The latest updates can be retrieved from
%  http://www.mathworks.com/matlabcentral/fileexchange/22022-matlab2tikz-matlab2tikz
%where you can also make suggestions and rate matlab2tikz.
%
\definecolor{mycolor1}{rgb}{0.00000,0.44700,0.74100}%
\definecolor{mycolor2}{rgb}{0.85000,0.32500,0.09800}%
%
\begin{tikzpicture}

\pgfmathsetmacro{\xminpt}{\xmin - \binwidth / 2}
\pgfmathsetmacro{\xmaxpt}{\xmax + \binwidth / 2}

\begin{axis}[%
width=\imgwidth,
height=\imgheight,
scale only axis,
bar width=\binwidth,
xmin=\xminpt,
xmax=\xmaxpt,
% [NOTE] Wartość `xtick` należy dostosować do danych.
%xtick={ 50, 100, 150, 200, 250, 300, 350, 400},
xlabel style={font=\color{white!15!black}},
xlabel={$x$},
ymin=0,
ymax=\ymaxaxis,
ylabel style={font=\color{white!15!black}},
ylabel={\# of cases},
axis background/.style={fill=white},
title style={font=\bfseries},
title={\% of cases in group \#2},
xtick pos=left,
legend style={legend cell align=left, align=left, at={(0.1,0.97)}, anchor=north, draw=white!15!black}
]
\addplot[ybar stacked, fill=mycolor1, draw=black, area legend] table[col sep=tab,row sep=newline]{data/histogram_groupped_data1.tsv};
\addlegendentry{group 1}

\addplot[ybar stacked, fill=mycolor2, draw=black, area legend] table[col sep=tab,row sep=newline]{data/histogram_groupped_data2.tsv};
\addlegendentry{group 2}

\addplot[forget plot, color=white!15!black] table[row sep=crcr, col sep=comma] ;

\node[above, align=center]
at (axis cs:0.350,4) {75\%};

\node[above, align=center]
at (axis cs:0.450,6) {67\%};

\node[above, align=center]
at (axis cs:0.550,33) {39\%};



\end{axis}
\end{tikzpicture}%
	\caption{\label{fig:histogram_groupped}A groupped histogram.}
\end{figure} 

\subsection{Spacing}

% SEC: podział strony (\newpage, \pagebreak)

Aby dokonać podziału stron w taki sposób, aby jej zawartość została równomiernie rozłożona na całej wysokości strony, skorzystaj z polecenia \lstinline|\pagebreak|. Polecenie \lstinline|\newpage| pozostawia pusty obszar u dołu strony.

\par\bigskip

% SEC: Lengths and when to use them

Zapoznaj się z zasadami stosowania poszczególnych typów poleceń wstawiających puste przestrzenie (np. \lstinline|\hspace|, \lstinline|\hspace*| itd.): \href{https://tex.stackexchange.com/a/41488/44391}{Lengths and when to use them (Stack Overflow)}.

\par\bigskip

% SEC: \<xxx>skip

Czasem warto dodać nieco światła między akapitami -- w tym celu skorzystaj z kombinacji polecenia \lstinline|\par| i jednego z poleceń z grupy \lstinline|\<small/med/big>skip|, np.:

\par\smallskip

\noindent
Lorem ipsum dolor sit amet\textellipsis 

\noindent
Lorem ipsum dolor sit amet\textellipsis (bez \verb|\par\smallskip|)

\par\smallskip

\noindent
Lorem ipsum dolor sit amet\textellipsis (po \verb|\par\smallskip|)


\subsection{Tips \& tricks}

W przypadku, gdy potrzebujesz wykonać pewną niestandardową operację wielokrotnie (czyli więcej niż \textbf{raz}) -- np. użyć nazwy przestrzeni barw \mbox{CIE $\textnormal{L}^{*}\textnormal{a}^{*}\textnormal{b}^{*}$}: \\
\hspace*{3em}\lstinline|CIE $\textnormal{L}^{*}\textnormal{a}^{*}\textnormal{b}^{*}$| \\
lub wygenerować wykresy identycznego typu, tyle że ze zmienionymi danymi (np. jak na Rys.~\ref{fig:scatter_plots}), wygodnie jest zdefiniować w preambule odpowiednie polecenie za pomocą polecenia \lstinline|\newcommand|.

\begin{figure}[H]
	\setlength{\imgwidth}{5cm}
	\setlength{\imgheight}{5cm}

	\centering
	
	\begin{subfigure}[c]{0.45\linewidth}
		% [NOTE] Polecenie `\PlotTDHist` zostało zdefiniowane przez użytkownika.
		\PlotTDHist{scatter1}{data/scatter1.csv}
	\end{subfigure}
	\begin{subfigure}[c]{0.45\linewidth}
		\PlotTDHist{scatter2}{data/scatter2.csv}
	\end{subfigure}
	
	\caption{\label{fig:scatter_plots}Wiele wykresów punktowych korzystających z tych samych parametrów konfiguracyjnych.
	}
\end{figure}

\par\bigskip

Warto oznaczać miejsca, w których coś jeszcze należy zmienić, za pomocą odpowiednik znaczników ,,TODO'' -- np. z użyciem polcenia \lstinline|\todo| z pakietu \lstinline|todonotes|, np.: tu\todo[inline]{Do poprawy.} należy coś poprawić.

\par\bigskip

Wygodnie jest zdefiniować ,,puste'' polecenie, które będzie służyło do wstawiania ,,komentarza'' w tekście -- w tym szablonie służy do tego \lstinline|\ignore|. Przykładowo, kod \lstinline|abc\ignore{xx} def| wyświetlany jest jako ,,abc\ignore{xx} def''.


\subsection{Bibliografia}

Zakres numerów stron zapisuj z użyciem półpauzy (zob. rozdz.~\ref{sub:typesetting}), czyli w postaci \lstinline|p1--p2| (za pomocą dwóch minusów).

\section{Results}

\section{Conclusions}

% SEC: sekcje nienumerowane

% [NOTE] Zwróć uwagę, że sekcję "Acknowledgments" wstawia się zwyczajowo bez numeracji - w tym celu należy stosować polecenie "z gwiazdką" (<typ_sekcji>*).
\section*{Acknowledgments}

This work was supported by the Dummy University based on the decision number XXX/YY.

% -----------

% SEC: \appendix
% [NOTE] Polecenie `\appendix` zmienia styl numeracji sekcji na zgodny ze stylem appendiksów (tj. na duże litery łacińskie).
\appendix

\section{SECTION 1}

\section{SECTION 2}

% -----------

Aby zmusić \LaTeX-a do umieszczenia w danym miejscu w tekście wszystkich ilustracji ze schowka i dopiero później umieszczenie kolejnego fragmentu tekstu \textbf{bez wstawiania podziału strony} użyj polcenia \lstinline|\afterpage{\clearpage}| z pakietu \lstinline|afterpage| (inaczej może się okazać, że ilustracja ze strony 3 znajdzie się na stronie 5\textellipsis).

\afterpage{\clearpage}  % pakiet: afterpage

% -----------

\printbibliography

\end{document}

