%------------------------------------------------------------------------------
% Rozmiary "obszarów roboczych", skład tekstu.
% SEC: rozmiary "obszarów roboczych", skład tekstu
%------------------------------------------------------------------------------

\usepackage{geometry}  % ustawienie rozmiaru marginesów, nagłówka i stopki itp.
\geometry{lmargin=20mm, rmargin=20mm, tmargin=15mm, bmargin=20mm, foot=10mm, head=10mm}

\usepackage{indentfirst}  % domyślne wcinanie pierwszego akapitu w bloku tekstu

%------------------------------------------------------------------------------
% Kodowanie znaków, lokalizacja.
% SEC: kodowanie znaków, lokalizacja
%------------------------------------------------------------------------------

\usepackage[utf8]{inputenc}  % wybierz kodowanie UTF-8
\usepackage{mathptmx}  % używaj czcionki Times jako domyślnej dla tekstu
\usepackage[T1]{fontenc}  % umożliwia wstawianie znaków z kodowania 8-bitowego (domyślnie jest 7-bitowe, przez co część znaków musi być wstawiana za pomocą kombinacji np. \l{} - "ł", \"{o} - "ö"; zob. https://tex.stackexchange.com/a/677/44391)
\usepackage[english]{babel}  % ustaw angielską lokalizację

%------------------------------------------------------------------------------
% Podstawowe pakiety.
% SEC: podstawowe pakiety
%------------------------------------------------------------------------------

\usepackage{color}  % STD
\usepackage[table,usenames,dvipsnames]{xcolor}

\usepackage[inline,shortlabels]{enumitem}  % zwiększa kontrolę użytkownika nad środowiskami `itemize` i `enumerate` (następca pakietu `enumerate`)

%------------------------------------------------------------------------------
% Tworzenie i konfiguracja tabel.
% SEC: tabele
%------------------------------------------------------------------------------

%\usepackage{hhline}
%\usepackage{longtable}
\usepackage{array}  % STD
\usepackage{tabularx}  % STD
\usepackage{multirow}  % umożliwia wstawienie tekstu zajmującego kilka wierszy [\multirow] - wymagane przez pakiet `makecell`
\usepackage{booktabs}  % poprawia estetykę tabel [\toprule, \midrule, \bottomrule, ...]
\usepackage{makecell}  % estetyczne, konfigurowalne formatowanie nagłówków tabeli [\thead]; umożliwia scalanie komórek [\makecell]
\usepackage[flushleft]{threeparttable}  % tabela z przypisami

% Definicje nowych typów kolumn w oparciu o typ `X` z pakietu `tabularx` (zob. https://en.wikibooks.org/wiki/LaTeX/Tables#The_tabularx_package)
\def\tabularxcolumn#1{m{#1}}  % spraw, aby kolumna typu X korzystała z `m` (`\parbox[c]`) zamiast z `p` (`parbox[t]`)
\newcolumntype{C}[1]{>{\hsize=#1\hsize\centering\arraybackslash}X}  % kolumna z wyśrodkowaną zawartością, o zadanej proporcji szerokości (zob. przykład z macierzą pomyłek)
\newcolumntype{Y}{>{\centering\arraybackslash}X}  % kolumna z wyśrodkowaną zawartością, bez precyzowania proporcji szerokości (zob. przykład z macierzą pomyłek)

\newlength{\multirowverticaloffset}
\setlength{\multirowverticaloffset}{2mm}

%------------------------------------------------------------------------------
% LaTeX (techniczne).
% SEC: LaTeX (techniczne)
%------------------------------------------------------------------------------

\usepackage{keyval,xparse}  % definiowanie nowych poleceń

\usepackage{cprotect}  % Pozwala na użycie środowiska `verbatim` w podpisach ilustracji, w tytułach sekcji itd. [\cprotect\caption{...}, \cprotect\section{...}]

\usepackage{courier}  % umożliwia korzystanie z czcionki Courier

\usepackage{footnote}  % lepsza kontrola przypisów w stopce

\usepackage{filecontents}  % emulowanie tworzenia pliku z danymi (np. pod kątem generowania wykresów) z poziomu kodu LaTeX-a

%------------------------------------------------------------------------------
% Matematyka.
% SEC: matematyka
%------------------------------------------------------------------------------

\usepackage{mathtools}  % STD
\usepackage{amsfonts}  % STD
\usepackage{amsmath}  % STD
\usepackage{amsthm}  % STD

%------------------------------------------------------------------------------
% Symbole.
% SEC: symbole
%------------------------------------------------------------------------------

\usepackage{textcomp}  % \textdegree (symbol stopni poza środowiskiem matematycznym)
\usepackage{gensymb}  % symbole matematyczne: \degree, \celsius, \micro, ...

%------------------------------------------------------------------------------
% Odnośniki.
% SEC: odnośniki
%------------------------------------------------------------------------------

\usepackage{url}  % definiuje polecenie `\url` służące do wyświetlania m.in. adresów e-mail, odnośników URL i ścieżek do plików - z możliwością podziału długiego adresu (domyślnie LaTeX nie dzieli takich ścieżek)
\renewcommand\UrlFont{\rmfamily\itshape}  % wyświetlaj adresy URL za pomocą czcionki Times z italikami

\usepackage{nameref}  % umożliwia wstawianie odnośników do sekcji z wyświetlaną nazwą sekcji 

\usepackage{hyperref}  % funkcjonalność "klikalnych" odnośników w dokumentach PDF
%\usepackage[hidelinks]{hyperref}  % wersja bez ramek wokół odnośników

%------------------------------------------------------------------------------
% Ustawienia ilustracji i ich podpisów.
% SEC: ilustracje i podpisy
%------------------------------------------------------------------------------

\usepackage{graphicx}  % STD

\usepackage{float}  % STD

\usepackage[font=small, labelfont=bf, labelsep=period]{caption}  % formatowanie podpisów pod ilustracjami (mała czcionka, wytłuszczony "identyfikator" zakończony kropką - a nie dwukropkiem)
%\captionsetup[subfigure]{labelfont=md}  % możliwość konfiguracji podpisów pod zadanym typem ilustracji/tabel...
%\captionsetup{subrefformat=parens}  % spraw, aby domyślnie odnośniki do ilustracji w grupie otaczane były nawiasami okrągłymi
\captionsetup{margin=2cm, singlelinecheck=on}  % singlelinecheck=on - podpis zawsze wyśrodkowany, gdy mieści się w jednej linii

\usepackage{subcaption}  % tworzenie grup ilustracji [\subfigure]
%\usepackage{subfig}  % obecnie zaleca się używać pakietu `subcaption` [2018-11-10]
\usepackage{afterpage}  % wymuszenie wyświetlenia wszystkich ilustracji i kontynuowanie wyświetlania tekstu na tej samej stronie [\afterpage{\clearpage}] https://tex.stackexchange.com/questions/88657/clearpage-without-pagebreak

% The new graphbox package introduces a new (vertical) align key which can be used in \includegraphics options.
\usepackage{graphbox}

% --------------------------------------

\newlength{\groupwidth}  % szerokość środowiska `figure`
\newlength{\picwidth}  % szerokość obszaru pojedynczego środowiska `subfigure`
\newlength{\imgwidth}  % szerokość grafiki w `subfigure`
\newlength{\imgheight}  % wysokość grafiki w `subfigure`

\newlength{\subfigshsep}  % odstęp między ilustracjami w poziomie
\newlength{\subfigshsepdef}  % odstęp między ilustracjami w poziomie (predefiniowany)
\newlength{\subfigsvsep}  % odstęp między ilustracjami w pionie
\newlength{\subfigsvsepdef}  % odstęp między ilustracjami w pionie (predefiniowany)
% [NOTE] Użycie dwóch typów długości - o wartościach modyfikowanych na bieżąco
%	i o wartościach predefiniowanych - pozwala zachować spójność dokumentu.

\setlength{\subfigshsepdef}{0.2cm}
\setlength{\subfigsvsepdef}{0.3cm}

% Domyślnie `figure` zajmuje 90% szerokości tekstu.
\setlength{\groupwidth}{0.9\textwidth}

% Konfiguracja środowiska `fbox` (do wyświetlania ramek)
\setlength{\fboxsep}{0pt}
\setlength{\fboxrule}{0.5pt}

%------------------------------------------------------------------------------
% Ustawienia parametrów odstępów i rozmiarów dla tytułów (pod)rozdziałów, czcionki, bibliografii, ilustracji i list - pozwalają "scieśnić" dokument.
% SEC: odstępy
%------------------------------------------------------------------------------

\usepackage{titlesec}

% Zmniejsz odstępy wokół list i między punktami listy.
\setlist{topsep=2pt,itemsep=0.1ex,partopsep=1ex,parsep=0.5ex,leftmargin=0.75cm}
%\setlength{\itemindent}{-1.5cm}

\newlength{\interpointgap}
\setlength{\interpointgap}{0.1em}

\setlength\parindent{0.4cm}

% Środowisko `enumerate` wymaga osobnej konfiguracji...
\setlist[enumerate]{topsep=2pt,itemsep=-1ex,partopsep=1ex,parsep=0.5ex}

% Zdefiniuj "ciasne" środowisko `enumerate` - z minimalnymi odstępami.
\newenvironment{tight_enumerate}{
\begin{enumerate}
	\setlength{\topsep}{4pt}
	\setlength{\itemsep}{4pt}
	\setlength{\parskip}{0pt}
}{\end{enumerate}}

% --------------------------------------

% Zmniejsz odstępy między literami.
%\usepackage[letterspace=-40]{microtype}

% Zmniejsz interlinię.
%\linespread{0.9}\selectfont

% --------------------------------------

% Zmniejsz odstęp między wpisami bibliograficznymi.
%\setlength\bibitemsep{0.4\itemsep}  % tylko dla pakietu `biblatex`

% --------------------------------------

% Zmniejsz odstępy między ilustracją i tekstem.
\setlength{\textfloatsep}{3pt}

% Zmniejsz odstęp między dwiema ilustracjami.
\setlength{\floatsep}{3pt}

% Zmniejsz odstępy powyżej i poniżej ilustracji.
\setlength{\intextsep}{3pt}

% Zmniejsz odstępy między grafiką a podpisem.
\setlength{\abovecaptionskip}{3pt}

% --------------------------------------

% Zmień formatowanie tytułów (pod)rozdziałów.

\titleformat{\section}{\bfseries\large}{\thesection}{1em}{}
\titlespacing{\section}{0pt}{1em}{0.5em}

%\titleformat{\subsection}{\bfseries\normalsize}{\thesubsection}{1em}{}
%\titleformat{\subsubsection}{\bfseries\normalsize}{\thesubsubsection}{1em}{}

%------------------------------------------------------------------------------
% Configure TikZ/PGFplots
% SEC: TikZ / PGFplots
%------------------------------------------------------------------------------

\usepackage{tikz}
\usepackage{pgfplots}
\usepackage{tikzscale}

\usetikzlibrary{calc,shapes,shadows,arrows,positioning,graphs}
\tikzset{>=latex}

\pgfplotsset{compat=1.8}
\pgfplotsset{minor grid style={dashed,black!15!white}}	
\pgfplotsset{major grid style={black!35!white}}	

\usepgfplotslibrary{groupplots}

%------------------------------------------------------------------------------
% Configure siunitx
% SEC: siunitx
%------------------------------------------------------------------------------

\usepackage{siunitx}  % wyświetlanie sformatowanych liczb (m.in. z jednostkami)

\sisetup{%
	range-phrase = {--},  % używaj półpauzy do oznaczenia zakresu
	range-units = single,  % nie powtarzaj jednostek przy zakresach
	group-digits = true,  % grupuj cyfry tysiącami
	group-separator = {,},  % separator grup zgodny z zapisem w USA
	output-decimal-marker = {.},  % separator części ułamkowej zgodny z zapisem w USA
}

\DeclareSIUnit\pixel{px}  % zdefiniuj własną jednostkę - piksel

%------------------------------------------------------------------------------
% Ustawienia bibliografii.
% SEC: bibliografia
%------------------------------------------------------------------------------

\usepackage[
	backend=biber,
	style=numeric,
	sorting=none,
	%
	% Zastosuj styl wpisu bibliograficznego właściwy językowi publikacji.
	language=autobib,
	autolang=other,
	%
	urldate=iso8601,
	% Nie dodawaj numerów stron, na których występuje cytowanie.
	backref=false,
	isbn=true,
	url=false,
	%
	% Ustawienia związane z polskimi normami dla bibliografii.
	maxbibnames=3,
]{biblatex}

\addbibresource{pub_sample_bibliography.bib}  % dodaj literaturę z zadanego pliku
\AtEveryBibitem{\clearfield{note}}  % usuń zawartość pól `note` z pozycji bibliograficznych

\usepackage{csquotes}
% NOTE: Ponieważ `csquotes` nie posiada polskiego stylu, można skorzystać z mocno zbliżonego stylu chorwackiego.
\DeclareQuoteAlias{croatian}{polish}

% Zmień sposób formatowania kombinacji "tom + numer + elektroniczne ID" (przydatne przy niektórych czasopismach).
%\renewbibmacro*{volume+number+eid}{%
%	\printfield{volume}%
%	\setunit{\addcomma\space}%
%	\printfield{number}%
%	\setunit{\addcomma\space}%
%	\printfield{eid}}

%------------------------------------------------------------------------------
% Konfiguracja pakietu `fp`
% SEC: fp
%------------------------------------------------------------------------------

\usepackage{fp}

% [NOTE] Polecenie `\FPuse` służy do wyliczania wartości wyrażenia "w miejscu" (tj. żeby od razu użyć wyliczonej wartości np. w innym środowisku).
\newcommand\FPuse[1]{\FPeval{\result}{#1}\result}


% [NOTE] Zdefiniuj polecenie `\numperc` służące do wyświetlania danego ułamka jako procentu (z zadaną liczbą miejsc po przecinku).
% Przykłady:
%	\numperc{1} => 100%
%	\numperc[precision=1]{0.3541} => 35.4%
\makeatletter
%
% [NOTE] Zdefiniuj klucz, z którego będzie korzystało makro.
%	\define@key{family}{key}{...}
%	Defines a macro \KV@prefix@key with one argument. When used in a keyval list, the macro receives the value as its argument.
\define@key{numprec}{precision}{\def\mm@precision{#1}}
%
\DeclareDocumentCommand{\numperc}{O{} +m}{%
	\begingroup%
	% [NOTE] Ustaw domyślną wartość klucza `first`
	\setkeys{numprec}{precision={0},#1}%
	%  
 	\FPeval{\result}{#2 * 100}%
 	{\num[round-mode=places,round-precision=\mm@precision]{\result}\%}%
 	%
	\endgroup%
}
\makeatother

% Wyświetl liczbę z zadaną dokładnością (stałą dla całego dokumentu).
\newcommand{\dispstat}[1]{\num[round-mode=places,round-precision=2]{#1}}

%------------------------------------------------------------------------------
% Debuggowanie.
% SEC: debugging
%------------------------------------------------------------------------------

% Facilitate the conditional compilation
\usepackage{etoolbox}  % umożliwia kompilację warunkową [\newtoggle, \toggletrue, \togglefalse, \iftoggle, ...]

\newtoggle{GFXDEBUG}  % zdefiniuj flagę `GFXDEBUG`
%\togglefalse{GFXDEBUG}
\toggletrue{GFXDEBUG}  % nadaj fladze `GFXDEBUG` wartość `true`

% Wyświetlenie grafiki z opcją `draft` spowoduje wyświetlenie ramki zamiast faktycznego obrazka - co przyspiesza "robocze" generowanie dokumentu:
%   \includegraphics[draft]{...}
% Po zdefiniowaniu polecenia, którego wartość wynosi "draft" w przypadku generowania "roboczego" i "" w przypadku generowania "normalnego" i umieszczeniu go w każdym poleceniu `\includegraphics`, możemy w efektywny sposób zmieniać sposób generowania dla całego dokumentu:
%   \includegraphics[\draftgraphics]{...}
\iftoggle{GFXDEBUG}{
	% Debug ON
	\newcommand{\draftgraphics}{draft}
}{
	% Debug OFF
	\newcommand{\draftgraphics}{}
}

%------------------------------------------------------------------------------
% Konfiguracja listingów (fragmentów kodu).
% SEC: listings
%------------------------------------------------------------------------------

\usepackage{listings}

\definecolor{mygreen}{RGB}{28,172,0}
\definecolor{mylilas}{RGB}{170,55,241}
\definecolor{mycolor1}{rgb}{0.000,0.502,0.502}
\definecolor{mycolor2}{rgb}{0.502,0.000,0.502}
\definecolor{Green}{rgb}{0.1,0.5,0.1}
\definecolor{MyDarkGreen}{rgb}{0.0,0.4,0.0}

\lstloadlanguages{TeX}%

% #include-s etc.
%	morecomment=[l][\color{magenta}]{\#}

%\lstset{%
\lstdefinestyle{mytex}{%
	language=TeX,
	frame=single,                           %  Single frame around code
	basicstyle=\small\ttfamily,             %  Use small true type font
	keywordstyle=[1]\color{Blue}\bfseries,  %  MATLAB functions bold and blue
	keywordstyle=[2]\color{Purple},         %  MATLAB function arguments purple
	keywordstyle=[3]\color{Blue}\underbar,  % User functions underlined and blue
	identifierstyle=,                       % Nothing special about identifiers
	% Comments small dark green courier
	commentstyle=\usefont{T1}{pcr}{m}{sl}\color{MyDarkGreen}\small,
	stringstyle=\color{Purple},             % Strings are purple
	showstringspaces=false,                 % Don't put marks in string spaces
	tabsize=2,
	%
	%%% Put standard language functions not included in the default
	%%% language here
	morekeywords={override},
	%
	%%% Put language function parameters here
	morekeywords=[2]{},
	%
	%%% Put user defined functions here
	morekeywords=[3]{},
	%
	deletekeywords={},
	%
	morecomment=[l][\color{Blue}]{...},     % Line continuation (...) like blue comment
	numbers=none,
	%        numbers=left,                           % Line numbers on left
	firstnumber=1,                          % Line numbers start with line 1
	numberstyle=\tiny\color{Blue},          % Line numbers are blue
	stepnumber=5,                           % Line numbers go in steps of 5
	%  breaklines=true,
	literate={ą}{{\k{a}}}1
           {ć}{{\'c}}1
           {ę}{{\k{e}}}1
           {ó}{{\'o}}1
           {ń}{{\'n}}1
           {ł}{{\l{}}}1
           {ś}{{\'s}}1
           {ź}{{\'z}}1
           {ż}{{\.z}}1
           {Ą}{{\k{A}}}1
           {Ć}{{\'C}}1
           {Ę}{{\k{E}}}1
           {Ó}{{\'O}}1
           {Ń}{{\'N}}1
           {Ł}{{\L{}}}1
           {Ś}{{\'S}}1
           {Ź}{{\'Z}}1
           {Ż}{{\.Z}}1,	
}

\lstset{style=mytex}

% Spraw, aby asterysk w czcionkach o stałej szerokości (np. Courier) był wyświetlany na wyższym poziomie.
\makeatletter
\lst@CCPutMacro
\lst@ProcessOther {"2A}{%
	\lst@ttfamily 
	{\raisebox{2pt}{*}}% used with ttfamily
	\textasteriskcentered}% used with other fonts
\@empty\z@\@empty
\makeatother

%------------------------------------------------------------------------------
% Varia.
% SEC: VARIA
%------------------------------------------------------------------------------

\usepackage{authblk}  % umożliwia podanie afiliacji autorów, gdy różni z różnych jednostek naukowych [\author, \affil]

\newcommand{\Lab}{\mbox{CIE $\textnormal{L}^{*}\textnormal{a}^{*}\textnormal{b}^{*}$}}

% Maksymalna "głębokość" numeracji.
\setcounter{secnumdepth}{4}

% Polecenie umożliwiające wstawienie "komentarza" w tekście (który nie będzie wyświetlany).
\newcommand{\ignore}[1]{}

\usepackage{todonotes}  % m.in. znacznik `\todo`

% Ręcznie zdefiniowany znacznik TODO.
%\newcommand{\todo}[1]{\noindent\colorbox{red}{\textcolor{white}{\bf\textsf{TODO}:}}~\textcolor{red}{#1}}
